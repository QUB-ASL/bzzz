\documentclass{article}

\usepackage{amsmath,amsfonts,amssymb,mathtools}
\usepackage[margin=1.2in]{geometry}
\usepackage{graphicx} 
\usepackage{float}
\usepackage{pgfplots}
\pgfplotsset{width=10cm,compat=1.9}

\title{Kalman Filter for Quadcopter Position Hold}
\author{Peter Davidson and Pantelis Sopasakis}


\begin{document}
\maketitle

\section{Problem statement}
The system dynamics is described by Newton's second law of motion.
Let $z$ be the altitude of the quadcopter, $g$ denotes the acceleration
due to gravity and $a$ is the acceleration that results from the forces
of the propellers. Then, the system dynamics is as simple as
Equation \eqref{eq:newton}
\begin{equation}
    \ddot{z} = a - g.
    \label{eq:newton}
\end{equation}
The acceleration, $a$, is an affine function of the thrust reference signal
(which comes from the RC); it is
\begin{eqnarray}
    a = \alpha \tau + \beta,
\end{eqnarray}
where $\alpha$ and $\beta$ are coefficients to be estimated; we can obtain
\textit{a priori} estimates offline and update them online using measurements
while flying (using the Kalman filter).


@Peter,
\begin{enumerate}
    \item Copy here the system dynamics from Chandra's report (we don't need Section 2.2.3)
    \item Write down what sensors we use and what their characteristics are (level of noise,
          presence of outliers, whether the sensors are biased, update frequencies)
\end{enumerate}
and we'll take it from there.



\newpage
\section{Altitude Dynamics}
The altitude dynamics of a quadcopter are defined within a global coordinate system, crucial for maintaining a predetermined altitude from the Earth's surface. The model that describes these dynamics is based on fundamental principles, delineated as follows:

The rate of change of the quadcopter's altitude, represented as \( \dot{z}_t \), is the result of the vertical acceleration \( a^z_{T_t} \) produced by the drone's motors at a given time minus the gravitational acceleration, \( g \). This equation is continuous in time and is expressed as,
\begin{equation}
\dot{z}_t = a^z_{T,t} - g
\end{equation}
\noindent
Here, \( a^z_{T_t} \) signifies the upward acceleration generated by the propulsion at time \( t \), measured in meters per second squared. The constant \( g \) denotes the acceleration due to Earth's gravity, also in meters per second squared. The altitude \( z_t \) represents the drone's center of mass's vertical position at time \( t \), measured in meters.

Additionally, the drone's vertical velocity \( v_{z_t} \) and vertical acceleration \( a_{z_t} \) are defined by the rate of altitude change \( \dot{z}_t \) and the rate of vertical acceleration change \( \dot{a}^z_{T_t} - g \), respectively. The term \( \dot{a}^z_{T_t} \) is derived from the quadcopter's upward thrust and serves as the system's input, while \( g \) is considered a constant input in the opposite direction.

Let \( y_{t}^z = z_t \) be the output equation of the system. 
Then the corresponding state-space representation is,
\begin{equation}
    \begin{bmatrix}
    v_{v,t}\\
    a_{t}^z
    \end{bmatrix} =
    \begin{bmatrix}
    0 & 1 \\
    0 & 0 
    \end{bmatrix}
    \begin{bmatrix}
    z\\
    v_{v,t}
    \end{bmatrix} + 
    \begin{bmatrix}
    0 & 0 \\
    1 & 1 \\ 
    \end{bmatrix}
    \begin{bmatrix}
    a_{T,t}^z \\ 
    -g
    \end{bmatrix},
    \label{eq:4}
\end{equation}

\noindent
for the barometer sensor's output, denoted by \( y_{barom} \), is described by the equation,
\begin{equation}
y_{barom} = z + d^{bar} + e_{barom}
\end{equation}
where, 
\begin{equation}
    d^{bar}_{t+1} = d^{bar}_t + w_{t}^{d^{bar}}
\end{equation}
The bias $(d^{bar})$ should stay consistent throughout readings, 
the second reading of the bias should be equal to the first allowing for some additional noise/offest.

\noindent
for the GPS sensor's output, 
denoted by \( y_{gps} \), is described by the equation,
\begin{subequations}
    \begin{align}
        y^{\rm gps} 
        {}={}&
        z + e^{\rm gps} 
        \\
        {}={}&
        \begin{bmatrix}
            1 & 0
        \end{bmatrix}
        x + e^{\rm gps} 
    \end{align}
\end{subequations}

The Time-of-Flight (ToF) sensor's output, denoted by \( y_{ToF} \), is described by the equation,
\begin{equation}
y^{\rm ToF} = z + d^{\rm ToF} + e_{\rm ToF}
\end{equation}


Throught all sensors \( y \) represents the output from the sensor. The variable \( z \) signifies the quadcopter's altitude, which is the measurement for all sensors. The term \( e \) encapsulates the measurement noise or errors associated with the sensors. This noise term, \( e \), encompasses various factors such as sensor inaccuracies, the impact of environmental conditions on sensor performance, and any systematic bias that might be inherent in the sensor's readings.

Equation \eqref{eq:4} describes the continuous-time altitude dynamics of the quadcopter. 
The discretization of the altitude dynamics of the system \eqref{eq:4} with a sampling frequency of Ts using the zero-order hold technique is,

\begin{equation}
    \begin{bmatrix}
    z_{t+1}\\
    v_{z,t+1}
    \end{bmatrix} =
    \begin{bmatrix}
    1 & T_s \\
    0 & 1 
    \end{bmatrix}
    \begin{bmatrix}
    z_t\\
    v_{s,t}
    \end{bmatrix} + 
    \begin{bmatrix}
    {1/2}{T^2}_s & {1/2}{T^2}_s \\
    T_s & T_s \\ 
    \end{bmatrix}
    \begin{bmatrix}
    a_{T,t}^z \\ 
    -g
    \end{bmatrix},
\end{equation}
\subsection{State Vector Definition}
The state vector \( x_t \) is defined as,
\begin{equation}
    x_t = 
    \begin{bmatrix}
        z_t &
        e_{t} & 
        \alpha_t & 
        \beta_t &
        d{_t}^{\rm bar}&
        d{_t}^{\rm ToF}&
    \end{bmatrix}
\end{equation}

\subsection{State Transition Equation}
\begin{equation}
    {x}_{t+1} = A_t {x}_t + W_t
\end{equation}

\subsection{Measurement Model}
The measurement model is defined as
\begin{equation}
y_t = C_t x_t + e_t
\end{equation}
\begin{equation}
    y_t  = 
    \begin{bmatrix}
        1 & 0 & 0 & 0 & 0 & 0 \\
        1 & 0 & 0 & 0 & 0 & 1 \\
        1 & 0 & 0 & 0 & 1 & 0 \\
    \end{bmatrix}
        x_t + e_t
\end{equation}
We may need to estimate \(\alpha\) or \(\beta\) for simplisity.

\section{Estimator Design}
A sentence
\begin{align}
    z_{t+1} =& z_t + T_s v_t^z + \frac{1}{2} T_s^2 (\alpha_t \tau_t + \beta_t) + w_t^z (\rm not In Use)\\
    z_{t+1} =& z_t + {T_s}{v_t} + w^{z}_{t}\\
    v_{t+1}^z =& v_t^z + T_s (\alpha_t \tau_t + \beta_t) + w_t^v\\
    \alpha_{t+1} =& \alpha_t + w_t^\alpha,\\
    \beta_{t+1} =& \beta_t + w_t^\beta\\
\end{align}
Consider using quation 14 for grater accuracy

where, 
\begin{equation}
    a_t =
    {\alpha_t}{\tau_t} + \beta_t
\end{equation}
We define $w_t^z$ and $w_t^v$ as the process noise elements.
$w_t^z$ is distributed normally with zero mean and variance
$\sigma_z^2$, expressed as $w_t^z \sim \mathcal{N}(0, \sigma_z^2)$, 
and similarly, $w_t^v$ follows a normal distribution with $w_t^v \sim \mathcal{N}(0, \sigma_v^2)$.
Additionally, $w_t^\alpha$ and $w_t^\beta$ represent white noise processes with distributions
$w_t^\alpha \sim \mathcal{N}(0, T_s \sigma_\alpha^2)$ and $w_t^\beta \sim \mathcal{N}(0, T_s \sigma_\beta^2)$ 
respectively. The system's state needing estimation is denoted by $x_t = (z_t, v_t^z, \alpha_t, \beta_t)$, 
with the system's dynamic model being describable as follows,
\begin{equation}
    x_{t+1} = A_t x_t + w_t^z,
\end{equation}

% altitude:
% \begin{equation}
%     z_{t+1} = z_t + {T_s}{v_t} + w^{z}_{t}
% \end{equation}

% velocity:
% \begin{equation}
%     v^{z}_{t+1} = v^{z}_t + T_s({\alpha_t}{\tau_t} + \beta_t) + w^{v}_t
% \end{equation}
% where, 
% \begin{equation}
%     a_t =
%     {\alpha_t}{\tau_t} + \beta_t
% \end{equation}


% State Transition Matrix (A)
The state transition matrix \( A \) describes how the state at time \( t \) evolves to the state at time \( t+1 \). For the given system, \( A \) is defined as:

Given the state vector \( x_t = \begin{bmatrix} z_t&  v_t^z&  \alpha_t&  \beta_t \end{bmatrix} \), the state transition matrix \( A_t \) from the system's dynamic model is defined as:
\begin{equation}
A_t = 
\begin{bmatrix}
1 & T_s & T_s \tau_t & T_s \\
0 & 1 & T_s \tau_t & T_s \\
0 & 0 & 1 & 0 \\
0 & 0 & 0 & 1
\end{bmatrix}
\end{equation}
where \( T_s \) is the sampling time, and \( \tau_t \) represents the throttle signal at time \( t \).

% Process Noise Covariance Matrix (Q)
The process noise covariance matrix \( Q \) represents the covariance of the process noise, accounting for the uncertainty in the model dynamics:
\begin{equation}
Q = 
\begin{bmatrix}
\sigma_z^2 & 0 & 0 & 0 \\
0 & \sigma_v^2 & 0 & 0 \\
0 & 0 & T_s \sigma_\alpha^2 & 0 \\
0 & 0 & 0 & T_s \sigma_\beta^2
\end{bmatrix}
\end{equation}
Here, \( \sigma_z^2 \), \( \sigma_v^2 \), \( \sigma_\alpha^2 \), and \( \sigma_\beta^2 \) represent the variances of the altitude, velocity, and the coefficients \( \alpha \) and \( \beta \), which relate the throttle signal to the lift.

% Measurement Matrix (C)
The measurement matrix \( C \) links the state vector to the measurement vector:
\begin{equation}
C_t = 
\begin{bmatrix}
1 & 0 & 0 & 0 & 0 & 0 \\
1 & 0 & 0 & 0 & 0 & 1 \\
1 & 0 & 0 & 0 & 1 & 0 \\
\end{bmatrix}
\end{equation}
This matrix considers the direct measurement of altitude by all sensors and accounts for biases in the ToF and barometer sensors.

% Measurement Noise Covariance Matrix (R)
The measurement noise covariance matrix \( R \) accounts for the uncertainty in sensor measurements:
\begin{equation}
R = 
\begin{bmatrix}
\sigma_{barom}^2 & 0 & 0 \\
0 & \sigma_{gps}^2 & 0 \\
0 & 0 & \sigma_{ToF}^2
\end{bmatrix}
\end{equation}
where \( \sigma_{barom}^2 \), \( \sigma_{gps}^2 \), and \( \sigma_{ToF}^2 \) are the variances of the measurement noises for the barometer, GPS, and Time-of-Flight sensors, respectively.

\section{Relationship between throttle signal and lift}
\begin{figure}[H]
    \centering
    \begin{tikzpicture}
        \begin{axis}[width=3in, height=1.5in, xlabel=Throttle signal, ylabel=Lift]
            \addplot table[x=Throttle signal, y=Lift, col sep=comma] {data.csv}; 
        \end{axis}
    \end{tikzpicture}
    \label{fig:fig1}
    \caption{}
\end{figure}
This graph represents the relationship between throttle signal 
From our radio, and how this effects the lifing force on the quad.
(please note that this graph is not to scale)
This relationship is catagorised by the equation, 
\[
\alpha_t\tau_t + \beta_t
\]
\section{Kalman Filter equations}
We are using 3 sensors for the altitude control namely the barometer, GPS, and the time of flight sensor. The barometer is known to have the least precision and has a significant bias. The Kalman filter equations for a sensor with bias is given as,
\begin{align}
    z_{t+1}=&\begin{bmatrix}
        A_{Bar}&\\
        &I
    \end{bmatrix}
        z_t^{Bar}
    +
        \Tilde{w}_t \\
    y_t=&\begin{bmatrix}
        C_{Bar}&I
    \end{bmatrix}
        x_t+v_t
\end{align}
$z_t^{Bar}=(x_t^{Bar},b_t)$, where $b_t$ defines the bias of the sensor and $x_t^{Bar}$ is the state. It is important to note that the above equation only suffices for the barometer which has a large bias. 

The other sensors which are the GPS and the time of flight do not have a great bias. So we can assume that they have the form of,
\begin{align}
    x_{t+1}=&Ax_t+w_t\\
    y_t=&Cx_t+v_t
\end{align}

So we can define our Kalman filter such that,
\begin{align}
    z_{t+1}=&\begin{bmatrix}
        B&&\\
        &ToF&\\
        &&GPS
    \end{bmatrix}z_t+\Tilde{w}_t \\
    y_t=&\begin{bmatrix}
        C_B&C_{ToF}&C_{GPS}
    \end{bmatrix}
        x_t+v_t
\end{align}
where,
\begin{align}
    B=&\begin{bmatrix}
        A_{Bar}&\\
        &I
    \end{bmatrix}\\
    ToF=&A_{ToF}\\
    GPS=&A_{GPS}
\end{align}
we also have $z_t=\begin{bmatrix}
   z_t^{Bar}&z_t^{ToF}&z_t^{GPS} 
\end{bmatrix}^\intercal,$  and $x_t=\begin{bmatrix}
    x_t^{Bar}&x_t^{ToF}&x_t^{GPS}
\end{bmatrix}^\intercal.$
\end{document}