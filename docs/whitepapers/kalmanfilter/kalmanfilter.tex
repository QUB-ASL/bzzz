\documentclass{article}
\usepackage{amsmath}
\usepackage[margin=1.2in]{geometry} 

\title{Kalman Filter for Quadcopter Position Hold}
\author{Peter Davidson and Pantelis Sopasakis}


\begin{document}
\maketitle

\section{Problem statement}
The system dynamics is described by Newton's second law of motion.
Let $z$ be the altitude of the quadcopter, $g$ denotes the acceleration
due to gravity and $a$ is the acceleration that results from the forces
of the propellers. Then, the system dynamics is as simple as
\begin{equation}
    \ddot{z} = a - g.
\end{equation}
The acceleration, $a$, is an affine function of the thrust reference signal
(which comes from the RC); it is
\begin{eqnarray}
    a = \alpha \tau + \beta,
\end{eqnarray}
where $\alpha$ and $\beta$ are coefficients to be estimated; we can obtain
\textit{a priori} estimates offline and update them online using measurements
while flying (using the Kalman filter).


@Peter,
\begin{enumerate}
    \item Copy here the system dynamics from Chandra's report (we don't need Section 2.2.3)
    \item Write down what sensors we use and what their characteristics are (level of noise,
          presence of outliers, whether the sensors are biased, update frequencies)
\end{enumerate}
and we'll take it from there.

% \section{Introduction}
% In this section, we delve into the application of Kalman filters for sensor fusion in quadcopter navigation, integrating data from GNSS modules, barometers, and time-of-flight sensors. The Kalman filter, a powerful algorithm for linear dynamic systems, it excels in estimating the state of a process by minimizing the mean squared error. It operates in two phases: prediction and update. During the prediction phase, it projects the current state and uncertainty forward in time. Then, in the update phase, it adjusts the projected estimate by incorporating new measurements, thereby refining the accuracy. By leveraging the Kalman filter, our aim is to enhance positioning accuracy and reliability, utilizing the unique advantages of each sensor to achieve a cohesive and robust solution for improved quadcopter control and stability for postition hold.

% \section{Prediction phase}
% The prediction phase of the Kalman filter forecasts the system's future state by extrapolating current estimates using system dynamics. It predicts the quadcopter's next position and velocity, incorporating uncertainties and process noise to account for potential deviations. This step ensures continuous trajectory estimation, crucial for real-time navigation adjustments.
% \\

% \noindent
% The state prediction phase is fundamental to forecasting the system's future state based on its current status and inherent dynamics. This phase is comprised of two main components:

% \subsection{State Prediction}
% The future state, denoted as \( \widetilde{x}_{t+1} \), is predicted by applying the State Transition Matrix, \( A_t \), to the current estimated state, \( \widetilde{x}_t \), assuming the expected value of the process noise, \( w_t \), is zero. The prediction is thus given by:
% \[
% \widetilde{x}_{t+1} = A_t \widetilde{x}_t
% \]
% This equation reflects the natural progression of the system from one state to the next, devoid of external inputs.

% \subsection{Estimated Variance Prediction}
% The variance of the estimator for the next state, \( P_{t+1} \), is also predicted to represent the uncertainty in the state estimate. This prediction incorporates the Process Noise Covariance Matrix, \( Q_t \), and is expressed as:
% \[
% P_{t+1} = A_t P_t A_t^T + G_t Q_t G_t^T
% \]
% Here, \( G_t \) adjusts the process noise to the state space dimensions, and \( A_t \) forwards the current error covariance in time.
% \\

% \noindent
% These predictive steps ensure the Kalman Filter has a baseline forecast of the system's state, which is then refined with incoming measurements during the update phase, maintaining the filter's accuracy and reliability.

% \section{Update phase}

% The update phase incorporates the latest measurements to refine the state estimates. The key equations involved in this phase are as follows:


% \subsection{Kalman Gain Calculation}
% Calculate the Kalman Gain \( K_t \), which dictates the extent to which the predicted state is corrected based on the measurement:
% \[ K_t = P_{t} C^T (C_t P_t C^T + R)^{-1} \]
% where \( R_t \) is the measurement noise covariance matrix.

% \subsection{State Update}
% Update the state estimate with the new measurement by applying the Kalman Gain to the innovation:
% \[ \widetilde{x}_{t+1} = \widetilde{x}_{t} + K_t ( y_t + - C\widetilde{x}_t)\]

% \subsection{Covariance Update}
% Update the estimate's error covariance to reflect the decrease in uncertainty:
% \[ P_{t+1} = (I - K_t C) P_{t} \]
% where \( I \) is the identity matrix.


\end{document}